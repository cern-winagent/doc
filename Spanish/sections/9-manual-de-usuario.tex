\section{Manual de usuario} \label{sec:man}
    En esta sección se detallan los mecanismos de interacción entre el usuario y la aplicación, además de las condiciones necesarias para que esta pueda ser utilizada correctamente. Se debe tener en cuenta que varios de los ejemplos mostrados a continuación son realizados utilizando rutas relativas, es decir, el directorio actual de las demostraciones se corresponde con el de la aplicación. Dependiendo de la consola utilizada es posible que sea necesario indicar la ruta relativa utilizando ``.\textbackslash'' delante del comando a ejecutar o alternativamente se puede utilizar la ruta absoluta, por ejemplo ``C:\textbackslash{}winagent\textbackslash{}winagent.exe''. 

    \subsection{Requisitos mínimos}
        Para poder utilizar \textit{Winagent} no son muchos los requisitos necesarios. El sistema operativo mínimo necesario para su utilización es \textit{Windows 7}, en cuanto a las versiones de escritorio, y \textit{Windows Server 2008} para los servidores.
        
        Se debe tener en cuenta que para instalar el agente como servicio, es necesario tener permisos de \textit{administrador} en el sistema. Además, en caso de utilizar algún \textit{plugin} que tenga dependencias específicas, estas deberán estar presentes en el equipo cuando se lleve a cabo su ejecución.
        
    \subsection{Manual de instalación}
        \subsubsection{Estructura de los archivos}
            Los archivos del sistema deben estar organizados con una estructura concreta, esto permite que todos los \textit{plugins} puedan ser cargados correctamente, al igual que las dependencias y la configuración a ejecutar. A continuación se muestra un ejemplo de dicha estructura y una breve explicación de cada uno de los componentes que se encuentran en ella.
            
            \dirtree{%
            .1 winagent.
                .2 plugins.
                    .3 plugin 1.
                    .3 plugin 2.
                    .3 plugin 3.
                .2 tmp.
                    .3 archivo temporal 1.
                    .3 archivo temporal 2.
                .2 dependencia 1.
                .2 dependencia 2.
                .2 plugins.dll.
                .2 winagent.exe.
                .2 winagent-updater.exe.
                .2 config.json.
            }
            
            \begin{itemize}
                \item \textbf{winagent} \\
                    Directorio raíz donde se almacenan los archivos relacionados con el agente en sí, así como las dependencias relacionadas con los \textit{plugins} que lo necesiten.
                    
                \item \textbf{plugins} \\
                    Directorio donde se encuentran todos los \textit{plugins} de entrada y salida que se deseen configurar.
                    
                \item \textbf{tmp} \\
                    Directorio donde se almacenan de forma temporal los archivos que son descargados por el actualizador automático. Esto incluye los archivos relacionados con el agente, \textit{plugins} y \textit{hashes} para comprobar la integridad de los archivos descargados.
                    
                \item \textbf{plugin.dll} \\
                    Interfaz de la que dependen todos los \textit{plugins} de la aplicación, es utilizada por el agente durante el proceso de carga para detectar todos componentes que la implementan.
                
                \item \textbf{winagent.exe} \\
                    Ejecutable del agente, utilizado desde la \textit{CLI} y por el servicio una vez instalado.
                
                \item \textbf{winagent-updater.exe} \\
                    Ejecutable del sistema de actualización automática, es lanzado directamente por el agente en intervalos programados.
                
                \item \textbf{config.json} \\
                    Archivo de configuración donde se almacenan todos los ajustes por los que se rigen tanto el agente como el proceso de actualización automática durante sus ejecuciones.
                
            \end{itemize}
            
        \subsubsection{Instalación del servicio}
            Con la finalidad de que \textit{Winagent} pueda funcionar como un servicio, primeramente deberá ser instalado como tal. Este proceso requiere que el usuario que lo realiza cuente con permisos de administrador. Para llevar a cabo la instalación puede ser utilizada cualquiera de las herramientas disponibles en el sistema operativo o el verbo \textit{service} junto con la opción \textit{-{}-install} de la propia aplicación.
            
            A continuación se muestran varias formas de instalar \textit{Winagent} utilizando diferentes métodos.
            
            \begin{lstlisting}[style=batch, caption=Instalación el servicio utilizando el propio agente]
                > winagent.exe service --install
            \end{lstlisting}
            
            \begin{lstlisting}[style=batch, caption=Instalación el servicio mediante la utilidad \textit{InstallUtil}]
                > InstallUtil winagent.exe
            \end{lstlisting}
            
            \begin{lstlisting}[style=batch, caption=Instalación el servicio utilizando \textit{PowerShell}]
                > New-Service -Name "Winagent" -BinaryPathName "C:\winagent\winagent.exe"
            \end{lstlisting}
            
     \subsection{Manual de utilización}
        \subsubsection{CLI}
            A través de la interfaz de línea de comandos que brinda el agente se puede hacer uso de todas sus funcionalidades utilizando el verbo \textbf{\textit{run}}. Para ello existen dos métodos, indicando manualmente las opciones en el comando ejecutado o especificando un archivo de configuración donde se encuentren definidos todos los ajustes a emplear.
            
            Al utilizar el verbo \textbf{\textit{run}}, por defecto, se utiliza el \textit{plugin de entrada} ``updates'' y el \textit{plugin de salida} ``output'', que es el que permite que los resultados sean mostrados en la consola.
            
            \begin{lstlisting}[style=batch, caption=Ejecución por defecto]
                > winagent.exe run
            \end{lstlisting}
            
            Las opciones \textit{-{}-input/-i} y \textit{-{}-output/-o} sirven para especificar manualmente los \textit{plugins} que son ejecutados. El resultado obtenido al utilizar el comando anterior sería exactamente igual al del siguiente comando:
            
            \begin{lstlisting}[style=batch, caption=Ejecución de forma manual]
                > winagent.exe run -i updates -o console
            \end{lstlisting}
            
            Además, también se pueden especificar las opciones necesarias para cada uno de los \textit{plugins} que se desea ejecutar. En el siguiente ejemplo se muestra cómo imprimir en pantalla los datos con un formato de tabla, indicando un valor de ``outputType'' al \textit{plugin} de salida. Para esto se deben utilizar las opciones ``input-options'' y ``output-options'', pudiendo especificar varias separándolas mediante comas(,).
            
            \begin{lstlisting}[style=batch, caption=Ejecución de forma manual]
                > winagent.exe run -o console --output-option outputType:table
            \end{lstlisting}

            Para utilizar un fichero de configuración, simplemente se debe indicar la ruta a este. Una vez iniciado, el agente carga los \textit{plugins} definidos, los cuales son ejecutados en los intervalos especificados. Para terminar la ejecución basta con presionar cualquier tecla.
            
            \begin{lstlisting}[style=batch, caption=Ejecución utilizando un fichero de configuración]
                > winagent.exe run C:\winagent\config.json
            \end{lstlisting}
        
        \subsubsection{Servicio}
        
            Cuando el agente es utilizado como servicio, se ejecuta en segundo plano y utiliza los ajustes especificados en un fichero de configuración nombrado \textit{config.json} que se encuentre su misma carpeta.
            
            Además, \textit{Winagent} cuenta con una herramienta que permite administrar el servicio a través de una consola de comandos, utilizando el verbo \textbf{\textit{service}}. A continuación se describen los comandos disponibles y su funcionalidad.
            
            \begin{lstlisting}[style=batch, caption=Instalar el servicio]
                > winagent.exe service --install
            \end{lstlisting}
            
            \begin{lstlisting}[style=batch, caption=Desinstalar el servicio]
                > winagent.exe service --uninstall
            \end{lstlisting}
            
            \begin{lstlisting}[style=batch, caption=Iniciar el service]
                > winagent.exe service --start
            \end{lstlisting}
            
            \begin{lstlisting}[style=batch, caption=Detener el servicio]
                > winagent.exe service --stop
            \end{lstlisting}
            
            \begin{lstlisting}[style=batch, caption=Reiniciar el servicio]
                > winagent.exe service --restart
            \end{lstlisting}
            
            \begin{lstlisting}[style=batch, caption=Comprobar el estado del servicio]
                > winagent.exe service --status
            \end{lstlisting}
            
        \subsubsection{Archivo de configuración}
            En el archivo de configuración se deben establecer todos los ajustes con los que se desea ejecutar el agente. Este debe tener una estructura específica que incluye información utilizada por el actualizador, los lectores de eventos del sistema, los \textit{plugins} de entrada y salida, y las opciones con las que se ejecutarán estos. A continuación se muestra la estructura básica que debe tener este fichero, mientras que en el anexo \ref{anx:settings} se puede apreciar un ejemplo real con el contenido de un archivo que ya ha sido puesto en producción.
            
            % TODO: crear estilo para archivo de configuracion
            \begin{lstlisting}[style=csharp, caption=Fichero de configuración]
                {
                  "autoUpdates": {
                    "enabled": true,
                    "source": "gitlab / github",
                    "uri": "ReleaseURL",
                    "schedule": {
                      "seconds": 0,
                      "minutes": 0,
                      "hours": 0
                    }
                  },
                  "eventLogs": [
                    {
                      "name": "Application",
                      "outputPlugins": [
                        {
                            "name": "OutputPluginName",
                            "settings": {
                              "SettingName": "SettingValue"
                            },
                            "schedule": {
                              "seconds": 0,
                              "minutes": 0,
                              "hours": 0
                            }
                        }
                      ]
                    }
                  ],
                  "inputPlugins": [
                    {
                      "name": "InputPluginName",
                      "settings": {
                          "SettingName": "SettingValue"
                      },
                      "outputPlugins": [
                        {
                          "name": "OutputPluginName",
                          "settings": {
                            "SettingName": "SettingValue"
                          },
                          "schedule": {
                            "seconds": 0,
                            "minutes": 0,
                            "hours": 0
                          }
                        }
                      ]
                    }
                  ]
                }
            \end{lstlisting}
            
    \subsection{Manual de desarrollo}
        Una de las características principales de \textit{Winagent} es que permite el desarrollo de nuevos \textit{plugins} que puedan ser combinados con los ya existentes sin modificar la aplicación, permitiendo obtener diferente información y utilizarla en nuevos sistemas. A continuación se muestran dos ejemplos y se especifican los pasos necesarios para la creación de librerías que puedan ser utilizadas como \textit{plugins}.
        
        \begin{enumerate}
            \item Compilar la librería base de los \textit{plugins}. Puede ser encontrada tanto en GitHub como en GitLab.
            
            https://gitlab.cern.ch/winagent/plugin
            
            https://github.com/cern-winagent/plugin
            
            \item Utilizando \textit{Visual Studio}, crear una \textit{Biblioteca de Clases (.NET Framework)}
            
            \item Añadir la \textit{dll} compilada en el paso 1 como referencia al nuevo proyecto
            
            \item Crear una clase llamada \textit{I[nombre del plugin]} u \textit{O[nombre del plugin]} que implemente la interfaz \textit{IInputPlugin} o \textit{IOutputplugin} respectivamente.
            
            \item Nombrar el \textit{plugin} utilizando el atributo \textit{PluginName}.
            
            \item Implementar el método \textit{Execute}, donde se especifican las funciones que realizará el \textit{plugin}.
        \end{enumerate}
    
        \begin{lstlisting}[style=csharp, caption=Plugin de entrada]
            namespace plugin
            {
                /// <summary>
                ///     This class implements the custom attribute required
                ///     by the agent to be recognized as a plugin.
                /// </summary>
                [PluginAttribute(PluginName = "Updates")]
                public class IUpdates : IInputPlugin
                {
                    /// <summary>
                    ///     Runs the main functionality of the plugin.
                    ///     Normally used to get system information.
                    /// </summary>
                    /// <param name="settings">
                    ///     Settings to be used inside the plugin.
                    /// </param>
                    /// <returns>
                    ///     A string formated as JSON.
                    /// </returns>
                    public string Execute(JObject settings)
                    {
                        ...
            
                        return jsonstring;
                    }
                }
            }
        \end{lstlisting}
        
        \begin{lstlisting}[style=csharp, caption=Plugin de salida]
            namespace plugin
            {
                [PluginAttribute(PluginName = "Console")]
                public class OConsole : IOutputPlugin
                {
                
                    /// <summary>
                    ///     Runs the main functionality of the plugin.
                    ///     Normally used to show or send to another
                    ///     system the information received.
                    /// </summary>
                    /// <param name="settings">
                    ///     Settings to be used inside the plugin.
                    /// </param>
                    public void Execute(string data, JObject settings)
                    {
                        ...
                    }
                }
            }
        \end{lstlisting}
