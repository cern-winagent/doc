% incluiranse todas as conclusións de tipo técnico e persoal.

    Mantener el control de las actualizaciones, reportes de discos duros y otros reportes de sistema en más de mil servidores, es un problema al que se han enfrentado durante años los administradores del \textit{CERN}. Este \textit{Trabajo de Fin de Máster} ha intentado facilitar esta tarea, de forma tal que se abarcasen todas las casuísticas de interés, y proveyendo un mecanismo integrable para cubrir otros que puedan surgir en el futuro.
    
    Tras un largo proceso de desarrollo, este proyecto llega a su fin, y, pese a haber supuesto más tiempo del planificado debido a los diferentes cambios introducidos, se considera que todos los objetivos propuestos han sido alcanzados. Además, es interesante mencionar que estos cambios han ayudado a obtener un sistema mucho más sólido que ya se ha puesto en producción satisfactoriamente.
    
    En el plano personal, me siento satisfecho de haber realizado este proyecto que representa un gran paso de avance dentro del CERN. Llevar a cabo su desarrollo me ha permitido ampliar mis conocimientos ya que he utilizando \textit{.NET} y \textit{CSharp}, un lenguaje de programación en el que no tenía experiencia  hasta comenzar con este proyecto. Además, aunque haya sido un proyecto de desarrollo individual, me ha dado la oportunidad de colaborar con fantásticos compañeros, lo cual me ha ayudado a entender los problemas y necesidades de los sistemas que ahora dependen de esta aplicación y así poder encontrar la solución mas adecuada.
