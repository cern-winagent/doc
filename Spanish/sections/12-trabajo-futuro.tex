\section{Trabajo futuro}
% presentaranse posible ampliacións e traballos relacionados por facer.
        
    Aunque el objetivo planteado en este proyecto ha sido alcanzado y se ha conseguido un sistema al que le pueden ser añadidas una innumerable cantidad de funcionalidades mediante el uso de \textit{plugins}, no cabe duda de que aún puede ser mejorado, ya sea añadiendo nuevas funcionalidades en el propio agente, o mejorando las tareas que realiza para lograr una mejor eficiencia y brindar más control sobre el producto.
    
    A continuación se detallan las posibles mejoras que pueden ser añadidas al proyecto en un futuro.
            
    \subsection{Generador de configuraciones}
        El hecho de que el sistema sea ampliamente flexible hace que el usuario deba especificar en un archivo de configuración todos los ajustes necesarios para su ejecución. Debido a esto, la complejidad del archivo puede llegar a ser muy alta, por lo que el usuario está propenso a cometer algún error durante su creación, lo que llevaría a un fallo durante la ejecución de la aplicación.
        
        Para solventar este problema, sería muy útil implementar un sistema que permitiese crear los archivos de configuración de forma semi-automática. De esta manera, el usuario solo debería seleccionar los \textit{plugins} que desee y rellenar los campos necesarios para obtener el contenido \textit{.JSON} con una configuración libre de errores.
        
    \subsection{Soporte flexible para \textit{APIs} en el sistema de actualización automática}
        El sistema de actualización automática implementado brinda soporte para las \textit{APIs} de \textit{GitHub} y \textit{GitLab}. Desde estas se puede obtener la información necesaria sobre los ficheros a actualizar, sin embargo, está limitado a la estructura de las mismas, por los que son las únicas fuentes que pueden ser utilizadas.
        
        Se podría crear un mecanismo que tomara la estructura de la \textit{API} desde el fichero de configuración, de forma que el usuario pueda establecerla manualmente. Esto permitiría el uso de \textit{APIs} personalizadas, pudiendo acceder a cualquiera de sus elementos.

    \subsection{Paquete \textit{MSI}}
        El sistema desarrollado puede ser instalado y ejecutado como un servicio, además de que tiene la capacidad de ser actualizado de forma automática en períodos configurables. No obstante, el despliegue inicial de la aplicación se realiza mediante \textit{scripts}, cuyo objetivo es copiar todos los archivos y realizar las instalaciones pertinentes en cada uno de los servidores.
        
        Una solución a esta limitación sería la creación de un \textit{Instalador de Windows (.msi)} \cite{msi}, un paquete desplegable que contenga los archivos del sistema junto con todas sus dependencias, y realice su instalación de forma automática. Esto representaría un gran avance, ya que es un tipo de instalador que podría ser desplegado fácilmente en varios sistemas, mediante el uso de \textit{herramientas de gestión de configuración} \cite{configmanag} como \textit{Puppet} \cite{puppet}, o directamente desde \textit{Active Directory}, herramientas que son actualmente utilizadas en el \textit{CERN}.

    \subsection{Detectar cambios en la configuración}
        Dado que la configuración del agente es leída justo después de haber iniciado el sistema, cualquier cambio realizado en esta es ignorado hasta que el servicio sea reiniciado. Lo que implica que si se desea modificar, se deberá reiniciar el servicio de forma manual o esperar a que ocurra una actualización de alguno de los componentes.
        
        Para solucionar esta dificultad se podría crear un mecanismo que monitorice el archivo de configuración con el que fue ejecutado el sistema, y una vez detecte algún cambio en este, reinicie el servicio de forma automática.

