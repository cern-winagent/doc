\label{sec:int}
%haberá que incluír unha introdución ao problema e xustificación do
%traballo realizado. En caso de que o TFG integre ou desenvolva traballos feitos na
%actividade doutras materias da titulación, o/a estudante deberá especificar os devanditos
%traballos e materias nesta sección.*/

La Organización Europea para la Investigación Nuclear, más conocida por sus siglas en francés \textit{CERN}, cuenta con más de treinta mil trabajadores asociados. En sus instalaciones se realiza una gran variedad de actividades diferentes, desde administrativas hasta de planificación e investigación. Entre estas actividades se incluyen siete experimentos en el Large Hadron Collider (\textit{LHC}), el acelerador de partículas más grande del mundo, compuestos por \textit{ATLAS} y \textit{CMS}, donde se utilizan detectores de uso general con el objetivo de abarcar el mayor rango posible de investigación, \textit{ALICE} y \textit{LHCb}, que investigan fenómenos específicos mediante el uso de detectores especializados, \textit{MoEDAL} que está centrado en la búsqueda de una partícula hipotética llamada \textit{the magnetic monopole} mediante el uso de detectores desplegados cerca del \textit{LHCb}, y los más pequeños \textit{TOTEM} y \textit{LHCf}que se centran en el análisis de \textit{fordward particles}, protones o iones pesados que rozan entre sí en lugar de encontrarse frontalmente cuando los haces chocan. \cite{lhc}

Todos estos procesos consumen una gran cantidad de tiempo y recursos, además de las labores de mantenimiento que provocan que el \textit{LHC} esté inoperable por largos períodos de tiempo. Por este motivo son muy importantes las tareas de planificación, preparación y simulación, que permiten que los resultados obtenidos mediante la realización de estos experimentos físicos sean lo más precisos posibles. Para ello es necesario que los servicios utilizados por el personal estén disponibles la mayor parte del tiempo, no sufran degradaciones y que ante cualquier fallo reciban una respuesta inmediata, tanto para evitar, como para resolver una incidencia. Esto incluye, entre otros, la monitorización de fallos en el software, hardware y parcheado de nuevas vulnerabilidades en la seguridad.

En el departamento de \textit{Information Technology} (\textit{IT}) se lleva a cabo la administración de estos servicios y los servidores necesarios para esta tarea. Dentro de este departamento, el grupo \textit{Collaboration, Devices and Applications} (\textit{IT-CDA}) es el encargado de gestionar los servicios de infraestructura, los cuales están directamente relacionados con \textit{Windows}, llegando a gestionar alrededor de 1000 de estos servidores, tanto físicos como virtuales, que incluyen entre otras cosas, los servicios de correo, autenticación, hosting web, virtualización, servidores de licencias, \textit{terminal servers} y servidores personalizados para cualquier departamento, grupo o experimento que lo requiera. \cite{infraservicios,infrawindows}

Actualmente la monitorización de estos servidores es realizada mediante \textit{SNMP} o \textit{scripts} que se ejecutan periódicamente, tomando información tanto del estado del hardware como del software, y filtrando algunos de los \textit{logs} que deben ser analizados. Esta información filtrada es enviada mediante correo electrónico a la persona responsable del área a la que pertenece la información, quien debe revisarla manualmente para descubrir cualquier incidencia ocurrida.

Un ejemplo de esto es la monitorización de las unidades de almacenamiento colocadas en la máquinas físicas. Los discos duros están configurados como \textit{RAID 1} (ver anexo \ref{anx:raid1}), lo que permite tener cierto nivel de resistencia a fallos, sin embargo, ante el mal funcionamiento de uno de los dispositivos, este tiene que ser reemplazado inmediatamente, ya que en caso de fallar el siguiente se perdería toda la información.

El proceso de detección de fallos en los discos duros comienza con \textit{scripts} que filtran diariamente los \textit{logs} generados por los sistemas actuales de monitorización y eventos del sistema (\textit{Event Logs}), con palabras como "\textit{Fail}", "\textit{Failure}" o "\textit{Error}", en el anexo \ref{anx:hw-info} se muestra un fragmento de uno de estos filtros. Una vez la información es recibida por el responsable, debe proceder a revisarla y, en caso de encontrar algún fallo, este deberá crear un \textit{ticket} de soporte para garantizar el remplazo del disco.

Debido a que el procedimiento mencionado representa una gran cantidad de tiempo y esfuerzo para cada una de las personas que deban desempeñar dichas tareas, surge el planteamiento base para el desarrollo de esta aplicación, la creación de una herramienta que sea capaz de englobar estos procesos de recopilación de información y servirla a diferentes sistemas de monitorización donde sea requerida para tomar las medidas oportunas. Además, crear esta aplicación personalizada, frente a otras soluciones como las basadas en \textit{SNMP}, permite abarcar de forma sencilla información requerida específicamente por la organización, por ejemplo, la información referente a las actualizaciones de \textit{Windows}.

El desarrollo de este proyecto es llevado a cabo en paralelo con \textit{CHOPIN}, un sistema de monitorización desarrollado por \textit{Marta Pacuszka}, el cual es el encargado de gestionar toda la información recibida desde \textit{Winagent} y presentarla al usuario utilizando una interfaz amigable, donde se puedan distinguir claramente las acciones necesarias en cada uno de los servidores.

Cabe mencionar que este proyecto, pese a ser una solución a un problema interno del \textit{CERN}, no está limitado a sus requisitos, es de código abierto y puede ser implementado en cualquier institución que cumpla con las condiciones necesarias para su instalación, además de que permite, debido a su carácter modular, la adición de funcionalidades para obtener y enviar información de casos de uso particulares.
