% TODO: ESTO LO VOY A ELIMINAR

%\section{Gestión de datos e información} \label{sec:dat}
%% Xestión de datos e información: describiranse os métodos ou técnicas empregadas para xestionar tanto os datos coma o resto %de información relevante.
%    El sistema desarrollado no está pensado para almacenar la información recopilada a través de los \textit{plugins} de %entrada. Cada vez que estos datos son obtenidos, automáticamente son procesados por los \textit{plugins} de salida %configurados, es decir, son mostrados al usuario o enviados a un sistema remoto independiente.
%    
%    Sin embargo, para que el agente funcione correctamente como un servicio, es necesario mantener un archivo de configuración %donde se especifiquen los ajustes de la aplicación. El contenido de este archivo dependerá tanto de los \textit{plugins} %que se deseen ejecutar, como de las opciones incluidas en estos.
%    
%    %% referenciar anexo
%    Un ejemplo de uno de estos archivos puede ser encontrado en \ref{anexos}, mientras que 
%    
%    De los \textit{plugins} planificados inicialmente con este proyecto, solo la mitad de estos permiten el uso de opciones %que condiciones su funcionalidad. Dichos \textit{plugins} son \textit{Console} y \textit{RabbitMQ}, a continuación se %muestran dos fragmentos de configuración donde se especifican los ajustes de los mismos.
%    
%    \begin{lstlisting}[style=csharp, caption=Ajustes del plugin console]
%        {
%            "name": "Console",
%            "settings": {
%              "OutputType": "table"
%            }
%        }
%    \end{lstlisting}