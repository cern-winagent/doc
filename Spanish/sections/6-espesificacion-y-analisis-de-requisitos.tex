\section{Análisis} \label{sec:req}
% describiranse os requisitos necesarios, tanto funcionais como non funcionais. Incluiranse os aspectos máis relevantes correspondentes á análise do traballo realizado.
    
    En esta sección se describen las historias de usuario que componen el \textit{backlog} del producto a desarrollar. En estas se especifica qué es lo que quiere el usuario y la razón por la que lo necesita, además de los requisitos que debe cumplir para determinar que la historia de usuario se ha completado correctamente.
    \begin{table}[h!]
        \resizebox{\linewidth}{!}{
            \begin{tabular}{|p{0.1\textwidth}p{0.8\textwidth}|}
                \hline
                \multicolumn{2}{|c|}{\textbf{Sistema modular}}                                  \\
                \hline
                \textbf{Como}   & usuario                                                       \\
                
                \textbf{Quiero} & que el sistema sea modular                                    \\
                
                \textbf{Para}   & incorporar nuevas funcionalidades sin modificar el sistema    \\
                \hline
                \hline
                \multicolumn{2}{|c|}{\textbf{Pruebas de aceptación}} \\
                \hline
                \multicolumn{2}{|p{\textwidth}|}{Los módulos pueden ser ejecutados de forma independiente}            \\
                \multicolumn{2}{|p{\textwidth}|}{El sistema es capaz de reconocer y utilizar módulos desarrollados independientemente}  \\
                \hline
            \end{tabular}
        }
    \end{table}
    
    \begin{table}[h!]  
        \resizebox{\linewidth}{!}{  
            \begin{tabular}{|p{0.1\textwidth}p{0.8\textwidth}|}
                \hline
                \multicolumn{2}{|c|}{\textbf{Servicio}}                                         \\
                \hline
                \textbf{Como}   & usuario                                                       \\
                
                \textbf{Quiero} & ejecutar la aplicación como un servicio                       \\
                
                \textbf{Para}   & que se mantenga permanentemente en ejecución                  \\
                \hline
                \hline
                \multicolumn{2}{|c|}{\textbf{Pruebas de aceptación}} \\
                \hline
                \multicolumn{2}{|p{\textwidth}|}{El servicio se inicia con el sistema}                   \\
                \multicolumn{2}{|p{\textwidth}|}{El servicio no debe ser detenido ante el fallo de los \textit{plugins}}                                                               \\
                \hline
            \end{tabular}
        }
    \end{table}
    
    \begin{table}[h!]  
        \resizebox{\linewidth}{!}{  
            \begin{tabular}{|p{0.1\textwidth}p{0.8\textwidth}|}
                \hline
                \multicolumn{2}{|c|}{\textbf{CLI}}                                              \\
                \hline
                \textbf{Como}   & usuario                                                       \\
                
                \textbf{Quiero} & ejecutar la aplicación desde la consola de comandos           \\
                
                \textbf{Para}   & realizar ejecuciones manuales                                 \\
                \hline
                \hline
                \multicolumn{2}{|c|}{\textbf{Pruebas de aceptación}} \\
                \hline
                \multicolumn{2}{|p{\textwidth}|}{La aplicación puede recibir la configuración desde un archivo}       \\
                \multicolumn{2}{|p{\textwidth}|}{La aplicación puede recibir la configuración desde la propia consola}\\
                \multicolumn{2}{|p{\textwidth}|}{Permite gestionar el servicio}            \\
                \hline
            \end{tabular}
        }
    \end{table}
    
    \begin{table}[h!]  
        \resizebox{\linewidth}{!}{  
            \begin{tabular}{|p{0.1\textwidth}p{0.8\textwidth}|}
                \hline
                \multicolumn{2}{|c|}{\textbf{Configuración}}                                    \\
                \hline
                \textbf{Como}   & usuario                                                       \\
                
                \textbf{Quiero} & especificar la configuración en un archivo                    \\
                
                \textbf{Para}   & desplegar el sistema fácilmente en varios servidores          \\
                \hline
                \hline
                \multicolumn{2}{|c|}{\textbf{Pruebas de aceptación}} \\
                \hline
                \multicolumn{2}{|p{\textwidth}|}{Permite especificar el origen de las actualizaciones}            \\
                \multicolumn{2}{|p{\textwidth}|}{Permite especificar las los parámetros para los \textit{plugins} de entrada y salida}            \\
                \multicolumn{2}{|p{\textwidth}|}{Permite vincular \textit{plugins} de entrada con \textit{plugins} de salida} \\
                \multicolumn{2}{|p{\textwidth}|}{Permite vincular eventos del sistema con \textit{plugins} de salida} \\
                \hline
            \end{tabular}
        }
    \end{table}
    
    \begin{table}[h!]  
        \resizebox{\linewidth}{!}{  
            \begin{tabular}{|p{0.1\textwidth}p{0.8\textwidth}|}
                \hline
                \multicolumn{2}{|c|}{\textbf{EventLogs}}                                        \\
                \hline
                \textbf{Como}   & usuario                                                       \\
                
                \textbf{Quiero} & capturar eventos del sistema en tiempo real                   \\
                
                \textbf{Para}   & detectar errores en el servidor lo antes posible              \\
                \hline
                \hline
                \multicolumn{2}{|c|}{\textbf{Pruebas de aceptación}} \\
                \hline
                \multicolumn{2}{|p{\textwidth}|}{Los eventos del sistema son leídos en menos de 10 segundos}            \\
                \multicolumn{2}{|p{\textwidth}|}{Se pueden capturar diferentes tipos de eventos}            \\
                \hline
            \end{tabular}
        }
    \end{table}
    
    \begin{table}[h!]  
        \resizebox{\linewidth}{!}{  
            \begin{tabular}{|p{0.1\textwidth}p{0.8\textwidth}|}
                \hline
                \multicolumn{2}{|c|}{\textbf{Actualizador automático}}                          \\
                \hline
                \textbf{Como}   & usuario                                                       \\
                
                \textbf{Quiero} & que la aplicación se actualice automáticamente                \\
                
                \textbf{Para}   & tener siempre la aplicación actualizada en todos los servidores   \\
                \hline
                \hline
                \multicolumn{2}{|c|}{\textbf{Pruebas de aceptación}} \\
                \hline
                \multicolumn{2}{|p{\textwidth}|}{Se comprueba que no haya errores de transferencia}            \\
                \multicolumn{2}{|p{\textwidth}|}{No se detiene la ejecución del servicio por más de 10 segundos}            \\
                \multicolumn{2}{|p{\textwidth}|}{Tiene soporte para \textit{GitHub}}            \\
                \multicolumn{2}{|p{\textwidth}|}{Tiene soporte para \textit{GitLab}}            \\
                \hline
            \end{tabular}
        }
    \end{table}
    
    \begin{table}[h!]  
        \resizebox{\linewidth}{!}{  
            \begin{tabular}{|p{0.1\textwidth}p{0.8\textwidth}|}
                \hline
                \multicolumn{2}{|c|}{\textbf{Actualizaciones}}                                      \\
                \hline
                \textbf{Como}   & usuario                                                       \\
                
                \textbf{Quiero} & obtener información de las actualizaciones del Sistema Operativo  \\
                
                \textbf{Para}   & controlar los servidores desactualizados                      \\
                \hline
                \hline
                \multicolumn{2}{|c|}{\textbf{Pruebas de aceptación}} \\
                \hline
                \multicolumn{2}{|p{\textwidth}|}{Se obtiene el número de actualizaciones disponibles}            \\
                \multicolumn{2}{|p{\textwidth}|}{Se determina la la fecha de instalación de la última actualización}            \\
                \hline
            \end{tabular}
        }
    \end{table}
    
    \begin{table}[h!]  
        \resizebox{\linewidth}{!}{  
            \begin{tabular}{|p{0.1\textwidth}p{0.8\textwidth}|}
                \hline
                \multicolumn{2}{|c|}{\textbf{Latidos}}                                          \\
                \hline
                \textbf{Como}   & usuario                                                       \\
                
                \textbf{Quiero} & tener un sistema de latidos                                   \\
                
                \textbf{Para}   & saber cuando un servidor ha fallado                           \\
                \hline
                \hline
                \multicolumn{2}{|c|}{\textbf{Pruebas de aceptación}}                            \\
                \hline
                \multicolumn{2}{|p{\textwidth}|}{La información tarda menos de 5 segundos en ser generada}  \\
                \multicolumn{2}{|p{\textwidth}|}{La información incluye la fecha del último reinicio}       \\
                \hline
            \end{tabular}
        }
    \end{table}
    
    \begin{table}[h!]  
        \resizebox{\linewidth}{!}{  
            \begin{tabular}{|p{0.1\textwidth}p{0.8\textwidth}|}
                \hline
                \multicolumn{2}{|c|}{\textbf{Consola}}                                          \\
                \hline
                \textbf{Como}   & usuario                                                       \\
                
                \textbf{Quiero} & que la información sea mostrada en consola                    \\
                
                \textbf{Para}   & depurar error con comprobaciones manuales                     \\
                \hline
                \hline
                \multicolumn{2}{|c|}{\textbf{Pruebas de aceptación}}                            \\
                \hline
                \multicolumn{2}{|p{\textwidth}|}{Permite imprimir la información en formato de tabla}       \\
                \hline
            \end{tabular}
        }
    \end{table}
    
    \begin{table}[h!]  
        \resizebox{\linewidth}{!}{  
            \begin{tabular}{|p{0.1\textwidth}p{0.8\textwidth}|}
                \hline
                \multicolumn{2}{|c|}{\textbf{RabbitMQ}}                                         \\
                \hline
                \textbf{Como}   & usuario                                                       \\
                
                \textbf{Quiero} & que la información se envíe a un servidor RabbitMQ            \\
                
                \textbf{Para}   & poder consumirla con otros sistemas                           \\
                \hline
                \hline
                \multicolumn{2}{|c|}{\textbf{Pruebas de aceptación}} \\
                \hline
                \multicolumn{2}{|p{\textwidth}|}{Permite enviar datos a múltiples servidores}               \\
                \multicolumn{2}{|p{\textwidth}|}{Permite especifica la cola a la que se desea enviar la información} \\
                \hline
            \end{tabular}
        }
    \end{table}
    